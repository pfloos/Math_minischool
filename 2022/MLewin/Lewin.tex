\documentclass{article}
\usepackage{xcolor,microtype,amsthm,amscd,amsmath,amssymb,amsfonts,physics,siunitx}
\usepackage{mathpazo,ragged2e,hyperref}

\setlength{\parindent}{0pt}

\newtheorem*{theorem}{Theorem}
\newtheorem*{definition}{Definition}
\newtheorem*{corollary}{Corollary}
\newtheorem*{lemma}{Lemma}
\newtheorem*{example}{Example}
\newtheorem*{proposition}{Proposition}
\newtheorem*{remark}{Remark}

\newcommand{\ZZ}{\mathbb{Z}}
\newcommand{\RR}{\mathbb{R}}
\newcommand{\CC}{\mathbb{C}}
\newcommand{\II}{\mathbb{I}}
\newcommand{\NN}{\mathbb{N}}

\newcommand{\cH}{\mathcal{H}}
\newcommand{\cL}{\mathcal{L}}
\newcommand{\cE}{\mathcal{E}}

\newcommand{\Id}{\text{Id}}
\newcommand{\hK}{\Hat{K}}

\newcommand{\eps}{\epsilon}
\newcommand{\sig}{\sigma}
\newcommand{\la}{\lambda}

\renewcommand{\sp}[2]{\langle #1,#2 \rangle}
\newcommand{\T}[1]{#1^{\intercal}}

\title{Notes on Mathieu Lewin's lecture: ``Mathematical description of solids''}
\author{badly copied by Pierre-Fran\c{c}ois Loos}

\begin{document}

\maketitle

%%%%%%%%%%%%%%%%%%%%%%%%%%
\section{Spectral theory}
%%%%%%%%%%%%%%%%%%%%%%%%%%

A bit of history about the spectral theory in infinite dimension...
\begin{itemize}
  \item 1906: Hilbert studied
  $$ \int_0^1 K(x,y) f(y) dy - \la f(x) = g(x) $$
  $K$ (symmetric) and $g$ are given, and $f$ is unknown.
  You can play with $\la$.
  Let us pass to a Fourier series, i.e., $\hK(k,k') \text{ with } k \in 2\pi \ZZ $.
  He decided to study infinite matrices.
  \item In 1906 paper, he explained how to diagonalize, introduced the name "spectrum" and studied continuous spectrum.
  \item 1925-26: Born, Heisenberg, Jordan (learnt from Hilbert) and they were at the same place in G\"ottigen.
  \item 1926: Hilbert started to teach quantum mechanics and von Neumann arrives as a postdoc.
  \item 1927: Dirac tries to formulate quantum mechanics (with theory of transformations) using kernels and introducing the famous delta function $\delta(x-y)$.
  \item 1929-32: von Neumann invented the concept of Hilbert space and the theory of unbounded operators. 
  The main message of his work was that infinite matrices must be put to the trash, and one must rely on unbounded operators.
\end{itemize}

\begin{definition} 
  Let $\cH$ be an abstract Hilbert space.
  $\sp{}{}$ is a scalar product that is antilinear on the left and linear on the right
  $\sp{v}{v} > 0 \ \forall v \neq 0$ and $\sp{v}{w} \ge \overline{\sp{w}{v}}$.
\end{definition} 

\begin{example} 
  $\cH = \CC^d \text{ and } \sp{x}{y} = x^* y = \T{\bar{x}} y $
  $$ \ell^2(\ZZ) = \qty{ (c_j)_{j \in \ZZ} \colon \sum \abs{g}^2 < \infty } $$
  In the definition of the Hilbert space we require that $\cH$ be complete (i.e., there is no hole) 
  $$ (x_n) \in \cH \qq{such that} \sum_n \norm{x_n} = \sqrt{\sp{x_n}{x_n}} < \infty \Rightarrow \sum_n x_n \text{ is convergent} $$
  An operator is a linear function $A$ defined on a domain $D(A) \subset \cH$ dense in $\cH$
  $$ f \in D(A) \mapsto A f \in \cH $$
\end{example} 

\begin{definition} 
  An operator $A$ is said to be bounded if there exists a constant $C$ such that 
  $ \norm{A v} \le C \norm{v} \ \forall v \in D(A)$ $\Leftrightarrow$ $\norm{A v}/\norm{v} \le C$ $\Leftrightarrow$ $A$ is bounded on the unit sphere 
  $$ v_n \to v \text{ with } v \in D(A) $$
  $$\norm{A v_n - A v} = \norm{A(v_n - v)} \le C \norm{v_n -v} \to  0 $$
\end{definition} 

\begin{proposition} 
  Let $A$ be a bounded operator. 
  Therefore, it admits a unique extension to the whole space $\cH$ which is again bounded.
  A bounded operator is fully characterized by its infinite matrix $\sp{e_n}{A e_m}$ for any orthonormal basis $(e_n)$.
  Note that Hilbert is right to use infinite matrix if one looks at bounded operators (but not unbounded).
\end{proposition} 

\begin{proof} 
  Let $v \in \cH$ and $v_n \in D(A)$ with $v_n \to v$ and thus $\norm{v_n - v} \to 0$.
  One can assume $\norm{v_n - v} \le 1/4^n$.
  $v = \sum_{n=0}^\infty (v_{n+1} - v_n)$ with $v_0 = 0$ is a convergent series
  $$\norm{v_{n+1} - v_n} \le \norm{v_{n+1} - v} + \norm{v - v_n} \le 1/4^{n+1} + 1/4^n \le 1/2^n$$
  $Av$ has to be convergent and $\lim_{N\to\infty} A = v_N$
  $$A v_N = \sum_{n=0}^{N-1} A(v_{n+1} - v_n)$$
  $$ \sum \norm{A(v_{n+1} - v_n)} \le C \sum \norm{v_{n+1} - v_n} $$
  $Av$ exists and is complete. 
  It is given by $\sum A (v_{n+1} - v_n) \in \cH$.
\end{proof} 

\begin{definition} 
  The norm of a bounded operator is the smallest constant $C$
  $$ \norm{A} = \sup_{\substack{v \in \cH \\ \norm{v}=1}} \norm{Av} $$
  The set of bounded operator is also complete for this norm.
  Let $A$ with $\norm{A} \le 1$ $\Rightarrow$ $1 - A$ is invertible as $(1 - A)^{-1} = \sum_{n=0}^\infty A^n$.
  Note that the identity is simply written as $1$ in the following.
\end{definition} 

\begin{definition} 
  An operator $A$ defined on $D(A)$ is unbounded when it is not bounded i.e. 
  $$\sup_{\substack{v \in D(A) \\ \norm{v} = 1}} \norm{Av} = \infty$$
  Most operator arising in quantum mechanics are unbounded. 
  For example, the momentum, etc.
\end{definition} 

\begin{definition}[spectrum] 
  The resolvent set is the set of all complex numbers $z \in \CC$ for which the equation $(A - z) f = g$ 
  admits a unique solution $f$ for all $g \in \cH$ with $f$ depending continuously on $g$ 
  $\Leftrightarrow$ there exists an inverse $(A - z)^{-1} \colon \cH \mapsto D(A)$ which is bounded.
  \begin{align*}
  	A - z \colon D(A) & \xrightarrow{\phantom{(A-z)^{-1}}} \cH \\
                 D(A) & \xleftarrow{(A-z)^{-1}} \cH \\
  \end{align*}
  The complementary set in $\CC$ is called the spectrum denoted by $\sig(A)$.
\end{definition} 

\begin{definition} 
  $\la$ is an eigenvalue when $(A-\la) f = 0$ although $f \neq 0$ (we have no uniqueness for the eigenvalues).
\end{definition} 

\begin{definition} 
  Everything in the spectrum which is not an eigenvalue is called the continuous spectrum.
\end{definition} 

\begin{remark} 
  if $A$ is bounded then the spectrum of $A$ is also bounded.
  $\sig(A) \subset D(O,\norm{A})$ where $O$ is the disk centered at origin of radius $\norm{A}$.
  For $ \abs{z} > \norm{A}$ and $A - z = - z (-A/z + 1)$, $\norm{A/z} = \norm{A}/z < 1$, so it is invertible with inverse by $(1 - A)^{-1} = \sum_{n=0}^\infty A^n$.
\end{remark} 

\begin{lemma} 
  The resolvent set is "open". 
  $z_0 \in \rho(A)$ (resolvents set) then $D(z_0,\norm{(A - z_0)^{-1}}) \subset \rho(A)$.
  $$ A - z = A - z_0 + z_0 - z = (1 - (z-z_0)(A-z_0)^{-1}))(A-z_0) $$
  $$ \abs{z - z_0} \norm{(A-z_0)^{-1}} < 1 $$
\end{lemma} 

\begin{definition}[Self-adjointness]
  $A$ is a $d \times d$ matrix is hermitian if $\T{\bar{A}} = A$
  $$ \sp{x}{A y} = x^* A y = \sp{A x}{y} $$
\end{definition} 

\begin{definition}
  We say that $A$ is symmetric when $\sp{Ax}{y} = \sp{x}{Ay} \ \forall (x,y) \in D(A)$
\end{definition} 

\begin{example} 
  For $A = -i \dv{}{x}$ on $]0,1[$ and $\cH = L^2(]0,1[) = \qty{f \colon \int_0^1 \abs{f}^2 <\infty }$.
  Be definition, $\sp{f}{g} = \int_0^1 \overline{f}(x) g(x) dx$. 
  Let $D(A) = C^1([0,1])$, then
  $$ \sp{f}{Ag} - \sp{Af}{g} = \int \overline{f}(x) i g'(x) dx - \int_0^1 - i \overline{f'}(x) g(x) dx = -i \qty[ g(1) \overline{f}(1) - g(0) \overline{f}(0) ] $$
  This is not symmetric!
  \\
  For $D(A_2) = { f \in C^1([0,1]) \colon f(0) = f(1) = 0 }$ (Dirichlet).
  This is symmetric and $\sig(A_2) = \CC$.
\end{example} 

\begin{theorem}
  Let A be a symmetric operator. 
  Then, the spectrum of $A$, $\sig(A)$, is either 
  \begin{enumerate}
    \item $\CC$
    \item $\CC_+ = \qty{ z \colon \Im{z} \ge 0 }$ 
    \item $\CC_- = \qty{ z \colon \Im{z} \le 0 }$ 
    \item $\subset \RR$
  \end{enumerate}
\end{theorem}

\begin{definition} 
  We say that $A$ is self-adjoint when it is symmetric and its spectrum is real (4th case of the previous theorem).
\end{definition}

\begin{proof} 
  $A$ symmetric means (with $z = a + i b$)
  $$ \norm{(A-z) v}^2 = \norm{(A - a - i b) v}^2 = \sp{(A - a - i b)v}{(A - a - i b)v} = \norm{(A - a)v}^2 + b^2 \norm{v}^2 $$
  $$ \norm{(A-z)v} \ge \abs{b} \norm{v} $$
  If the inverse exists, then $\norm{w} \ge \abs{b} \norm{(A-z)^{-1} w}$ and $\norm{(A-z)^{-1}} \le 1/\abs{b}$.
  It is important to mention that $\norm{(A-z)^{-1}}$ is related to the inverse of $\abs{b}$.
\end{proof}

\begin{remark}
  What are the consequences of the theorem and definitions?
  \begin{enumerate}
    \item If $A$ is self-adjoint on a domain $D(A)$ then it cannot be self-adjoint on a smaller domain $D \subsetneq D(A)$ and $\sig(A_{|D}) = \CC$.
    Let $A$ such as $i \in \rho(A)$, $A - i$ is invertible from $D(A) \mapsto \cH$ and $\forall g \ \exists! f \ (A - i) f = g$ $\Rightarrow i \in \sig(A_{|D})$.
    \item If $A$ has $\sig(A) = \CC_{\pm}$ then no self-adjoint extension and $\sig(A_{|D}) = \CC$ if $D \subsetneq D(A)$.
    \item If $A$ is self-adjoint then "$D(A)$ is complete" $\Leftrightarrow$ "$A$ is closed".
    $v_n \in D(A) \to v$ and $A v_n \to w$ $\Rightarrow$ $v \in D(A)$ and $A v = w$.
    $A v_n = (A - i) v_n + i v_n$ and $v_n = (A - i)^{-1} [Av_n - i v_n]$, therefore $v = (A - i)^{-1} (w - i v) \in D(A)$.
  \end{enumerate}
\end{remark}

\begin{theorem}
  Let $A$ be a self-adjoint operator $D(A)$. 
  Then, $\la \in \RR$, $\la \in \sig(A)$ $\Leftrightarrow$ it exists a sequence $v_n \in D(A)$ with $\norm{v_n} = 1$ such that $(A - \la) v_n \to 0$ (quasi-eigenvalue).
\end{theorem}

\begin{proof} 
  ($\Leftarrow$) by contradiction. If $(A - \la)^{-1}$ exists, then it cannot be bounded 
  $1 = \norm{v_n} = \norm{(A - \la)^{-1} (A - \la) v_n} \le C \norm{(A - \la) v_n} \to 0$ (contradiction).
  \\
  ($\Rightarrow$) by contraposition
  $\nexists v_n$ such that 
  $$\inf_{\substack{v\in D(A) \\ \norm{v} = 1}} \norm{(A - \la)v} = C > 0$$
  Because $\norm{(A - \la) v} \ge C \norm{v} \ \forall v \in D(A)$, this implies $\la \notin \sig(A)$.
  $\norm{(A - \la - i \eps)v}^2 = \norm{(A - \la)v}^2 + \eps^2 \norm{v}^2 \ge (c^2 +\eps^2) \norm{v}^2$ 
  $\Rightarrow$ $\norm{(A - \la - i \eps)^{-1}} \le 1/C$ 
  $\Rightarrow$ $\la \notin \sig(A)$
\end{proof} 

\begin{example} 
  momentum $P$ on $]0,1[$ which fulfils $P f = - i f'$.
  What is the domain? 
  Note that we have to be able to solve $-i f' \pm i f = g \ \forall g \in L^2(]0,1[)$ that is defined as $(P \pm i)^{-1}$.
  Let $P_{min} f = - i f'$ with $D(P_{min}) = \qty{ f \in C^1 \colon f(0) = f(1) = 0 }$.
  Let $P_{max} f = - i f'$ with $D(P_{max}) = \qty{ f \in L^2 \colon f' \in L^2 }$.
  $P_{max}$ is not symmetric and $\sig(P_{max}) = \CC$ with only eigenvalues.
  Let $P_0 f = - i f'$ with  $D(P_0) = \qty{ f \in L^2 \colon f' \in L^2 f(0) = f(1) = 0 }$.
  $P_0$ is symmetric.
\end{example} 

\begin{lemma} 
  $\sig(P_0) = \CC$ has pure continuous spectrum.
  $- i f' = z f$ and $f(x) = C e^{ixz} \Rightarrow \sig(P_{max}) = \CC$ 
  but none of these solutions satisfy Dirichlet $\Rightarrow$ $P_0$ has no eigenvalue.
  For $g = e^{i\overline{z}x}$, we have 
  $$ \sp{g}{(P_0 - z)f} = \int_0^1 \overline{g}(t) [i f'(t) - z f(t)] dt = \int_0^1 \overline{(-i g'-\overline{z} g)}(t) f(t) dt = 0$$
  This implies $z \in \sig(P_0)$ 
  $P_0 - z$ cannot be "onto" (surjective): $\nexists f \in D(P_0)$ such that $(P_0 - z) f = g$ $\Rightarrow$ $\int_0^1 \abs{g}^2 = 0$, which is a contradiction.
  (bijection: one-to-one).
\end{lemma}

\begin{theorem} 
  The only self-adjoint realization of the momentum on $]0,1[$ is:
  $P_{per,\theta} f = - i f'$ (Born-vonKarman, BvK).
  $$ D(P_{per,\theta}) = \qty{ f \in L^2 \colon f' \in L^2, f(1) = e^{i \theta} f(0) } $$
  $\sig(P_{per,\theta}) = \theta + 2\pi \ZZ$ all of the eigenvalues with eigenfunctions $e^{i(\theta+2\pi k)x}$.
\end{theorem}

\begin{remark} 
  $D(P_0) \subset D(P_{per,\theta}) \subset D(P_{max}) \ \forall \theta$ 
\end{remark}

\begin{remark}
  $e_n(x) = \sqrt{2} \sin(\pi n x)$ with $n \ge 1$ 
  and $e_n(0) = e_n(1) = 0 \ \forall n$ with $e_n \in D(P_0)$
  $\Rightarrow$ the infinite matrix $\sp{e_n}{P_{per,\theta} e_m} = \sp{e_n}{P_0 e_m}$ is independent of $\theta$.
\end{remark}

\begin{theorem}[momentum on the real line (2.37 in ML)]
  There is only one self-adjoint realization of the momentum on $\RR$:
  $P f = - i f'$ with $D(P) = \qty{ f \in L^2 \colon f' \in L^2 }$. 
  The spectrum is $\sig(P) = \RR$ with no eigenvalue (pure continuous spectrum).
\end{theorem}

\begin{remark}
  $-i f' = k f$ has solution $f = C e^{ikx} \ \forall k \in \CC$. 
  When $k$ is real, $e^{ikx}$ is bounded it is not in $L^2$ ($\Rightarrow$ no eigenfunction).
  Let $f_n(x) = \chi(x/n) e^{ikx}$, we have
  $$\int_{-\infty}^{\infty} \abs{f_n}^2 = \int_{-\infty}^{\infty} \chi(x/n)^2 dx = n \int \chi^2$$
  $g_n = f_n/\norm{f_n}^2$ are quasi-eigenfunctions.
\end{remark}

\begin{proof} 
  We use Fourier: $\overline{P f}(k) = k \overline{f}(k)$
  with $D(P) = \qty{ f \in L^2 \colon f' \in L^2 } = \qty{ f \in L^2 \colon k \hat{f} \in L^2 }$.
  The only thing one has to do:
  $\forall g \in L^2 \ \exists! f \in D(P)$ such that $-i f' \pm i f = g$ and $(k+i) \hat{f} = \hat{g}$.
\end{proof}

\begin{theorem} 
  The only self-adjoint momentum on $\RR$ is
  $P f = -i f'$ with $D(P) = \qty{f \in L^2(\RR) \colon f' \in L^2(\RR) }$
  and $\sig(P) = \RR$ with no eigenvalue.
\end{theorem} 

\begin{proof} 
  We do Fourier again: $\overline{-i f'}(k) = k \hat{f}(k)$ "multiplication operator".
  For $z = -i$, we have $(P + i)f = g$, i.e. $(P + i)f = g$ $\Rightarrow$ $\hat{f} = \hat{g}/(k+i)$
  \begin{align*}
  	A - z \colon D(A) & \xrightarrow{\phantom{(A-z)^{-1}}} \cH \\
                 D(A) & \xleftarrow{(A-z)^{-1}} \cH \\
  \end{align*}

  Because Fourier is unitary, $g \in L^2 \Leftrightarrow \hat{g} \in L^2$
  and $\int \abs{g}^2 = \int \abs{\hat{g}}^2$.
  $D(P) = \qty{ f \in L^2 \colon \exists \hat{g} \in L^2 \hat{f} = \hat{g}/(k+i) } = \qty{ f \in L^2 \colon k \hat{f} = - i f' \in L^2 }$.
  The inverse is bounded: 
  $ \norm{(P + i)^{-1} g}^2_{L^2} = \norm{\hat{(P+i)^{-1}g}}^2_{L^2} 
  = \int \abs{\hat{g}(k)}^2/(k^2+1) dk \le \int \abs{\hat{g}(k)}^2 dk = \norm{g}^2_{L^2}$ with $C = 1$. 
  This implies that $-i \in \rho(P)$ and the same derivation stands for $z=+i$.
  $P$ is symmetric: $\sp{f}{Pg} = \int_\RR k \overline{\hat{f}(k)} \hat{g}(k) dk$
  $\Rightarrow$ $P$ is self-adjoint (idea: Fourier diagonalizes $P$).
\end{proof}

\begin{remark}
  For a potential $V$, $f \mapsto V f$, we have
  $$ D(V) = \qty{ f \in L^2 \colon V f \in L^2, \int V(x)^2 \abs{f(x)}^2 dx < \infty }$$
  and $\sig(V) = \Im(v)$.
\end{remark}

\begin{example}[momentum on half line $\RR_+ = [0,\infty[$]
  Result: it does not exist!
  $P_0 f = - i f'  D(P_0) = \qty{ f \in L^2 \colon f' \in L^2, f(0) = 0 }$
  and $\sig(P_0) = \CC_-$
\end{example}

\begin{example}[kinetic energy or laplacian]
  On $]0,1[$, there are many self-adjoint realizations (depending on the boundary conditions): Dirichlet, Neumann, periodic, etc.
  On $\RR$, there is one self-adjoint $D(-\dv[2]{}{x}) = \qty{ f \in L^2 \colon f'' \in L^2 }$.
  On $\RR^d$ with $d \in \qty{1,2,3}$, we have 
  $D(-\Delta) = \qty{ f \in L^2 \colon \Delta f \in L^2 } = \qty{ f \in L^2 \colon |k|^2 \hat{f} \in L^2 }$.
\end{example}

\begin{theorem}[Kato]
  For the Schr\"odinger operators $H = - \Delta + V$.
  Assume that $V$ is itself $L^2$ locally and uniformally, i.e.,
  $\int_{B(x_0,1)} V(x)^2 dx \le C \ \forall x_0 \in \RR^d$ [i.e., $V \in L^2_{unif}$]
  Then, $H = - \Delta + V$ is self-adjoint on $D(-\Delta)$.
\end{theorem}

\begin{example}[hydrogen atom for $d = 3$, $V(x) = -c/\abs{x}$]
  $$ \sup_{x_0} c^2 \int_{B(0,1)} \frac{dx}{\abs{x-x_0}^2} = c^2 \int_{B(0,1)} \frac{dx}{\abs{x}^2} < \infty $$
\end{example}

\begin{proof}
  Let $-\Delta + V +m^2$ with $m \gg 1$. We have
  $$ -\Delta + V +m^2 = (1 + V(-\Delta+m^2)^{-1} ) (-\Delta + m^2) $$
  One shows $\norm{V(-\Delta + m^2)^{-1}} < 1$ if $m \gg 1$.
  For $d = 3$, this operator has the kernel $K(x,y) = V(x) \frac{e^{-m\abs{x-y}}}{\abs{x-y}}$.
  $(Kf)(x) = \int K(x,y) f(y) dy$ and 
  $\norm{Kf}_{L^2} \le C_m \norm{f}_{L^2}$ and $C_m < 1$ when $m \gg 1$.
\end{proof}
  
\begin{example}
  Spectral theorem and continuous spectrum:
  \begin{itemize}
    \item
    $P_{per,0}$ (the periodic momentum) is diagonal in the basis $e_k = e^{2i\pi kx}$ ($k \in \ZZ$)
    $$ 1 = \sum_{k \in \ZZ} \dyad{e_k} $$
    $$ P_{per,0} = \sum_{k \in \ZZ} 2\pi n \dyad{e_k} $$

    \item
    $P$ (momentum in $\RR$) is diagonal in Fourier $e_k(x) = e^{ikx}/\sqrt{2\pi}$
    $$ 1 = \int_\RR   \dyad{e_k} dk  \quad \text{(not well defined on $L^2$ but OK on nice functions)} $$
    $$ P  = \int_\RR k \dyad{e_k} dk $$  
    $f(x) = 1/\sqrt{2\pi} \int_\RR \sp{e_k}{f} e^{ikx} dk$ and
    $\hat{f}(k) = \sp{e_k}{f} =  1/\sqrt{2\pi} \int_{\RR} f(x) e^{-ikx} dx$

    \item
    For a self-adjoint operator $A$:
    $1 = \int \sum_n \dyad{\varphi_{E,n}} dE$ with ($\varphi_E \in ??$). 
    $dE$ has to be replaced by $d\mu(E)$ and the integration is on $\sig(A)$.
    And we also add multiplicities!
  \end{itemize}
\end{example}

\begin{example}[Laplacian in $\RR$]
  $$ A = - \dv[2]{}{x} = \int_\infty^\infty k^2 \dyad{e_k} dk \quad (E = k^2) $$
  $$ A = \int_0^\infty E \qty(\dyad*{e_{\sqrt{E}}} + \dyad*{e_{-\sqrt{E}}}) dE/(2\sqrt{E}) $$
\end{example}

\begin{theorem}
  We set $d = 3$ and $H = - \Delta + V$.
  We assume that $V$ is nice and decays fast (Coulomb not allowed).
  Then, there exists eigenfunctions $f_1,\cdots,f_N$
  and generalized eigenfunctions $\varphi_k(x)$, $k \in \RR^3$, $k^2 \notin \cE$ (a set of bad energies).
  The $\varphi_k$'s are bounded functions $\varphi_k(x) \sim e^{ikx}$ ($\abs{x} \to \infty$)
  and $(-\Delta + V)\varphi_k = k^2 \varphi_k$.
  $$ 1 = \sum_{n=1}^N \dyad{f_n} + \int_{\RR^d} \dyad{\varphi_k} dk $$
  $$ H = \sum_{\la_n \le 0} \la_n \dyad{f_n} + \int_{\RR^d} k^2 \dyad{\varphi_k} dk $$                                                                                      
  This result follows from scattering theory.
  Solve Lipmann-Schwinger equation
  $$ \varphi_k(x) = e^{ikx} - \frac{1}{4\pi} \int e^{ik\abs{x-y}} \frac{V(y)}{\abs{x-y}} \varphi_k(y) dy $$
  $(1 + K) \varphi_k = e^{ikx}$ (multiply by $\sqrt{V}$).
  See Reed-Simon vol.~3 Thm X1.14 ($V \in L^1$).
\end{theorem}

\begin{theorem}
  Assume that $d = \qty{1,2,3}$ and $\int_B(x_0,1) V(x)^2 dx \le C \ \forall x_0 \in \RR^d$ and
  $H = -\Delta + V$ and $D(H) = D(-\Delta)$.
  Then there exists a Borel measure $\mu$ on $\RR$ and a function 
  $N \colon \RR \mapsto \NN$ ($\mu$-mesurable) and functions $\varphi_{E,n}$ for $\mu$ almost every $E$  with $n=1,\ldots,N(E)$ such that 
  $$ 1 = \int \sum_{n=1}^{N(E)} \dyad{\varphi_{E,n}} d\mu(E) $$
  $$ H = \int E \sum_{n=1}^{N(E)}  \dyad{\varphi_{E,n}} d\mu(E) $$
  $(-\Delta+V)\varphi_{E,n} = E \varphi_{E,n}$ with $\abs{\varphi_{E,n}} \le C_E (1 + |x|^{d/2+\epsilon})$.
\end{theorem}

\begin{theorem}[Simon "Schr\"odinger semi-groups", 1982]
  If $\abs{\varphi(x)} \le C(1+\abs{x}^N)$ for any $N$ with $\varphi \neq 0$ 
  such that $(-D + V) \varphi = E \varphi$, 
  then $E \in \sigma(-\Delta+V)$.
  (Result from Schnol 1953, rediscovered by Simon in 1981.)
\end{theorem}

\begin{remark}
  $$1 = \int \sum_{n=1}^{N(E)} \dyad{\varphi_{n,E}} d\mu(E)$$ 
  This defines a unitary operator $L^2(\RR^d) \simeq \bigoplus L^2(\RR,d\mu)$
  with $f \in \Delta \colon \tilde{f}(n,E) = \sp{\varphi_{n,E}}{f}$
  $$\int_{\RR^d} \abs{f}^2 = \int \sum_{n=1}^{N(E)} \abs{\tilde{f}(n,E)}^2 d\mu(E) $$
\end{remark}

\begin{theorem}
  If $A$ is a self-adjoint operator on $\cH \subset \RR$, 
  then there exists a measure $\mu$ on $\RR$ and sets $(M_n)_{n \ge 1}$ that are decreasing and a unitary 
  $U \colon \cH \mapsto \bigoplus_{n \ge 1} L^2(M_n,d\mu)$
  such that $U A U^*$ is $\tilde{f}(n,E) \mapsto E \tilde{f}(n,E)$
  \begin{itemize}
    \item $N(E)$
    \item formal generalized eigenfunctions in the new picture which are $\delta_{E,n}$ "$\varphi_{n,E} = U^{-1} \delta_{E,n}$". 
  \end{itemize}
  General result by Poerschke-Stolz and Weidmann (1989).
  $\varphi_{n,E} \in D(T)'$ (prime is for dual) if $T$ is so that $f(a) T^{-1}$ is Hilbert-Schmidt (where $f$ is any continuous bounded function).
  $A = -\Delta + V$ with $T = (1+\NN^{d/2+\eps})$.
\end{theorem}

\begin{remark}
  When we have diagonalize an operator $A$, we can define a function of the operator $f(A) = \int f(E)$. 
  For example, $e^{itA} f(X) = e^{itX}$ with spectral projection: $f(x) = \II_{(\infty,E]} (x)$.
\end{remark}

%%%%%%%%%%%%%%%%%%%%%%%%%%%%%%%%%%%%%%%%
\section{One-particle periodic problem}
%%%%%%%%%%%%%%%%%%%%%%%%%%%%%%%%%%%%%%%%

We are interested by the equation $-\nabla + V(x)$ when $V$ is periodic.
\\

\begin{remark} 
  Let $V = 0$, and we consider $P$. 
  The group of translation $(\RR,+)$ is acting on $L^2(\RR)$.
  For $\tau \in \RR$, we have $x \mapsto x + \tau$. $U_\tau \colon f(x) \mapsto f(x-\tau)$.
  Formally at least, $U_\tau P U_{-\tau} = P$ (the domain of $P$ is invariant under all the $U_\tau$)
  $ \Rightarrow U_{\tau} e^{itP} U_{-\tau} = e^{itP}$ (see Chapter 4 of ML).
\end{remark} 

\begin{example}[Fourier]
    $f(x+\tau) = \la(\tau)f(x) \forall x, \forall \tau$\\
    $\la(\tau+\tau') = \la(\tau)\la(\tau')$ (it's a character)\\
    $\la(\tau) = e^{\alpha \tau}$\\
    $f(x+\tau) = e^{\alpha\tau} f(x) \ \forall x,\tau$\\
    $f(x) = C e^{\alpha x}$ with $\alpha \in i\RR$\\
\end{example}

\begin{example}[1-Periodic]
    subgroup $(\ZZ,+)$ (we only translate by integers)\\
    $f(x+j) = \la(j)f(x) \ \forall j \in \ZZ \forall x \in \RR$\\
    $\la(j+j') = \la(j)\la(j')$ $\Rightarrow$ $\la(j) = \la(0)^j = e^{\alpha j}$\\
    $f(x+j) = e^{\alpha j} f(x)$ with $\alpha = i \theta \in i\RR$\\
    $f(x+j) = e^{i\theta j} f(x) \ \forall j \in \ZZ$ and $\forall x \in \RR$  (this is the BvK condition)\\
\end{example}

\begin{example} 
  Let us consider $P$
  $$ 1 = \frac{1}{2\pi} \int \dyad{e_k}{e_k} dk
      = \frac{1}{2\pi} \int_{-\pi}^{\pi} \sum_{j\in\ZZ} \dyad{e_{\theta+2\pi j}{e_{\theta+2\pi j}}}   d\theta $$
  with $e_k = e^{ikx}$ and $- \dv[2]{}{x} + V$ with $V$ is 1-periodic.
  We slice with respect to $\theta$ and we fix $\theta \colon - \dv[2]{}{x} + V$ on functions $f(x+1) = e^{i\theta} f(x) \forall x$.
  These are not in $L^2(\RR)$.
  We solve $(-\Delta+V)f = \la f$ for fixed $\theta$
  $\Leftrightarrow$ solving on $]0,1[$ with the BvK $f(1) = e^{i\theta} f(0)$
  $f'(x+1) = e^{i\theta} f'(x) \ \forall x$ and 
  $f'(1) = e^{i\theta}f'(0)$ (extra condition for 2nd-order differential equation).
\end{example} 

\begin{theorem} 
  $\RR^d$ with $d=\{1,2,3\}$. 
  We consider a lattice $\cL = \sum_{i=1}^d v_i$ where $(v_1,...,v_d)$ is a basis.
  $C$ is the Wigner-Seitz cell (one possible choice of a unit cell) and is defined as $\qty{ x \in \RR \colon \abs{x} < \abs{x - k} \ \forall k \in \cL\setminus\{0\} }$.
  $\cL^*$ is the dual lattice equal to $2\pi\ZZ \sum_i^d v_i^*$.
  $B$ is the Brillouin zone.
  Let $V$ be a $\cL$-perodic potential. We assume $\int_C V(x)^2 dx < \infty$.
  $H = -\Delta + V$ is self-adjoint on $D(-\Delta)$.
  Let also $H(\xi) = -\Delta + V$ on $L^2(\CC)$ with BvK boundary condition [$\xi$ is a vector $(\xi_1,...,\xi_d)$].
  This operator has an infinite many eigenvalues $\la_n(\xi) \to + \infty$ as $ n\to\infty$ with eigenfunctions $\varphi_{n,\xi}$.
  $$ H = \frac{1}{\abs{B}} \int_B \sum_n \la_n(\xi) \dyad{\varphi_{n,\xi}} d\xi $$
  $$ 1 = \frac{1}{\abs{B}} \int_B \sum_n \dyad{\varphi_{n,\xi}} d\xi $$
  with $ \sig(H) = \cup_{n \ge 1} \la_n(B)$
\end{theorem} 

\begin{theorem} 
    $H = -\Delta +  V$ with $V$ a $\cL$-periodic pontential and bounded, i.e, $\int_C V^2 < \infty$
    We have $(-\Delta +  V) f = \la V$.
    The common generalized eigenfunctions of the $\cL$-translations are the functions which satisfies the BvK translations: 
    $f(x+\ell) = e^{i\xi\cdot\ell} f(x)$ for some $\xi \in B$ (where $B$ is the Brillouin zone, i.e., the unit cell of the dual lattice).
    We look for generalized eigenfunctions for fixed $\xi$.
    Thus, we can solve the equation only in the Brillouin zone.
    Therefore, we have only true eigencuntions with the BvK boundary condition.
    Let $H(\xi)$ be $-\Delta + V$ in $L^2(C)$ with the BvK boundary condition.
    The spectrum of $H(\xi)$ is 
    $\sig(H(\xi)) = \qty{ \la_1(\xi) \le \le_2(\xi) \le \cdots }$ and $\la_n(\xi) \to \infty$ when $n \to \infty$.
    which are continious functions of $\xi$.
    Let call $\varphi_{n,\xi}$ the generalized eigenfunctions.
    $$ \II_{L^2(\RR^d)} = \frac{1}{\abs{B}} \int_B \sum_n \dyad{\varphi_{n,\xi}} d\xi $$
    $$ H = \frac{1}{\abs{B}} \int_B \sum_n \la_n(\xi) \dyad{\varphi_{n,\xi}} d\xi $$
    $$ \sig(H) = \bigcup_{n \ge 1} { \la_n(\xi) \colon \xi \in \bar{B} } $$
    The union of the "Bloch" bands $\qty{ \la_n(\xi), \xi \in \bar{B} }$ is the spectrum.
\end{theorem}

\begin{proof} 
  See Chapter 7 in the ML book
\end{proof}

\begin{definition}[Fermi sea \& density of states]
  We put infinitely many electrons but non-interacting... Of course.
  Thermodynamic limit: 
  $-\Delta+V on \RR^d$
  We consider a very large cube with hard wall ($+\infty$ potential outisde the box).
  We put $N$ electrons on a box of length $L$: $N \sim \bar{\rho} L^d$ (with $\bar{\rho}$ fixed)
  Basically we use Dirichlet boundary condition but one can prove that the result would be the same with other boundary conditions such as periodic or Neumann.
  Hamiltonian for $N$ non-interacting electrons with no spin
  $$ H = \sum_j=1^N (-\delta + V)_{|C_L}(x_j) $$
  where $|C_L$ means that we are restricted to the cube of length $L$ (Dirichlet).
  The wave functions $\Psi$ are Slater determiants on the $N$ first eigenfunctions of $(-\Delta + V)_{|C_L}$
  $\rho = N/L^d$ with $N \to infty$ and $L^d \to \infty$
  We rather look at the one-particle density matrix.
  $\gamma_L$ is projection on the space spanned by the $N$ first eigenfunctions $L^2(C_L)$
\end{definition}

\begin{theorem}
  For $\bar{\rho}$ given, $\forall f \in L^2(\RR^d)$, the spectral projection is $\gamma_L \colon \II_{C_L} f \mapsto \II_{(-\infty,\eps_F]} (-\Delta+V)$, 
    where the Fermi energy $\eps_F$ is so that the density of state is correct 
    $$ \bar{\rho} = \frac{1}{(2\pi)^d} \int_B \sum_{n \ge 1} \II(\la_n(\xi) \le \eps_F) d\xi$$
    $\bar{\rho}$ is the integrated density of state.

    $$ \gamma_{\infty} = \frac{1}{\abs{B}} \II_{(-\infty,\eps_F}(-\Delta+V) 
                    = \frac{1}{\abs{B}} \int_B \sum_{n \ge 1} \II(\la_n(\xi) \le \eps_F) \dyad{\varphi_{n,\xi}} d\xi $$
    $\gamma_{\infty}$ is named the Fermi sea.
    $$ \rho_{\infty}(x) = \frac{1}{\abs{B}} \int_B \sum_{n \ge 1} \II(\la_n(\xi) \le \eps_F) \abs{\varphi_{n,\xi}}^2 d\xi $$
    and $\rho_{\infty}(x)$ is $\cL$-periodic.
    Often, one has an integed number of particle per cell, i.e., $K$ electrons per unit cell
    Therefore, $\Bar{\rho} = \frac{K}{\abs{C}}$. Therefore, 
    $$ K = \frac{1}{\abs{B}} \int_B \sum_{n \ge 1} \II(\la_n(\xi) \le \eps_F) d\xi $$
    Insulator: it costs something to add an electron.
    Metal: it costs nothing to add an electron (gapless).
    Due to Sommerfeld, Block and Bethe (at $T = 0$).
    Shape of the $\la_n(\xi)$ plays an important role in particular close to the Fermi level $\eps_F$.
\end{theorem}
    
\begin{definition}[Regularity of $\la_n(\xi)$ and degeneracies]
    $H(\xi) = -\Delta+V$ with BvK boundary condition. 
    By multiplying by $e^{i\xi \cdot x}$, we have
    $\Tilde{H}(\xi) = \abs{-i \nabla+\xi}^2 + V$ periodic boundary condition ($\xi = 0$)
    (it's like a magnetic gauge)

    $$ \Tilde{H}(\xi) = -\Delta + V + 2 P\cdot\xi + \abs{\xi}^2 $$
    The term $2 P\cdot\xi$ is a linear perturbation while $\abs{\xi}^2$ is a trivial shift.

    $$ A + \xi \cdot B = A + \sum_i \xi_i B_i $$
\end{definition}

\begin{example}[degeneracies for $2\times2$ matrices]
    For $d = 1$, 
    $$\begin{pmatrix}
      \xi_1 &     0   \\
      0    & -\xi_1 \\
    \end{pmatrix}$$
    Eigenvalues are $\pm\xi_1$.
    Even for a uniform matrix, the eigenvalues are not smooth.

    For $d = 2$,
    $$\begin{pmatrix}
      \xi_1 &  \xi_2 \\
      \xi_2 & -\xi_1 \\
    \end{pmatrix}$$
    Eigenvalues are $\pm\sqrt{\xi_1^2 + \xi_2^2}$ (Dirac cone).
\end{example}

\begin{theorem}
    We assume a regular (at least continuous) family of $N \times N$ hermitian matrices $A(\xi)$ (in $\xi \in \RR^d$). 
    $a,b \in \RR$ which are not in $\sig(A(\xi_0))$.
    Then $a,b$ are also not in $\sig(A(\xi))$ for $\xi \sim \xi_0$ and the spectral projection
    $$\II_{[a,b]}(A(\xi))  = \sum_{\la_n(\xi) \in [a,b]} \dyad{v_{n,\xi}}$$ 
    has the same regularity as $A$.
\end{theorem}
\begin{corollary}
  $\la_n(\xi_0)$ is multiplicity one, then $\la_n(\xi)$ is also multiplicity one for $\xi \sim \xi_0$ and has the same regularity as $A$.
\end{corollary}

\begin{proof}[proof of the corollary] 
  We choose $\la_n(\xi) \in [a,b]$.
  One can show that $\la_n(\xi) = \Tr[A(\xi)\II_{[a,b]}(A(\xi))]$ is nice!
\end{proof}  

\begin{proof}[proof of the theorem] 
    $A(\xi) - z = A(\xi_0) - z + A(\xi) - A(\xi_0) = (1 + (A(\xi) - A(\xi_0)) (A(\xi_0) - z)^{-1}) (A(\xi_0) - z)$
    We know that $A(\xi_0) - z$ is invertible for $z \sim a$ and $z \sim b$.
    We know also that $A(\xi) - A(\xi_0)$ is small by continuity arguments.
    Therefore, $A(\xi) - z$ is invertible.
    
    $(A(\xi) - z)^{-1} = (A(\xi_0) - z)^{-1} (1 + B)^{-1}$ (regularity of series).
    The resolvent is as nice as the operator (where it is defined).
\end{proof}  
 
\begin{definition}[Cauchy's formula] 
  $$\II_{[a,b]} = \frac{1}{2\pi} \oint_{C} \frac{dz}{z-x}
  \qq{$\Rightarrow$}  \II_{A(\xi)} = - \frac{1}{2\pi} \oint_{C} \qty[A(\xi) - z]^{-1} dz$$
\end{definition}

\begin{theorem}{One-dimensional degeneracies}
    if you order things properly, the eigenvalues are smooth (analytical actually).
    $A(\xi)$ is analytic in $\xi$ (i.e. the Taylor series is convergent).
    $\la = \la_n(\xi_0)$ is degenerate of multiplicity $m$.
    Then for $\xi \sim \xi_0$ the spectrum of $A(\xi)$ in the neighbourhood of $\la$ consistes of m analytic functions which coincide at $\xi_0$.
    Theorem due to Rellich. Proof in Reed-Simon vol 4 Thm XII.3.
\end{theorem}

\begin{remark}
  Analiticity is special. 
  For example, it is wrong for $C^2$ functions.
\end{remark}

%    
%  2D degeneracies
%  We look at family of matrices that depends on two parameters.
%  \Tilde{H}(\xi) = |-i \nabla+\xi|^2 + V
%  If \cL is invariant under some rotation R, \Tilde{H}(R\xi) = U_R \Tilde{H}(\xi) U_R^*
%  The Hamiltonian \Tilde{H}(\xi) is unitarily equivalent to $\Tilde{H}(R\xi)$.
%  The two Hamiltonians are unitarily equivalent is one rotates \xi as well.
%
%  Thm: A(\xi) \subset C^1 family of 2x2 hermitian matrices and \xi \in \RR^2.
%  I assume \exists \theta \in ]0,2\pi[\{\pi} and R is the associated rotation acting on \RR^2.
%  If assume A(R\xi) = U(\xi) A(\xi) U(\xi)^* \forall \xi
%  U(\xi) is continuous.
%  We assume that there is a degeneracy somewhere: A(0) = 0
%  Then, \exists unitary matrix V such taht A(\xi) = \delta V (...) V^* + o(|\xi|)
% 
%  ... = | \xi_1  \xi_2  |
%        | \xi_2  -\xi_1 |
%



%%%%%%%%%%%%%%%%%%%%%%%%
\section{Many-body case}
%%%%%%%%%%%%%%%%%%%%%%%%

%%%%%%%%%%%%%%%%%%%%%%%%
\section*{References}
%%%%%%%%%%%%%%%%%%%%%%%%

\begin{itemize}
  \item Mathieu Lewin, \textit{Th\'eorie spectrale et mécanique quantique}, Springer, 2022.
  \item Michael Reed \& Barry Simon (4 volumes): Functional Analysis, Fourier Analysis \& Analysis of Operators.
  \item Davies 1995 
\end{itemize}

\end{document}


